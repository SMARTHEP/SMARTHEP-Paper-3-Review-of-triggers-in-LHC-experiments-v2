\section*{Appendix}

\subsection{ATLAS detector}

The ATLAS detector\cite{ATLASMachine} is a general-purpose detector consisting of several sub detectors covering nearly all the area around the collision point. The detectors consist of the inner detector(ID), electromagnetic and hadronic calorimeters and a muon system. A small superconducting solenoid surrounds the inner detector with three large superconducting toroids(one around the barrel and two in the endcaps) around the calorimetry.

The inner detector surrounds the beam pipe, has a 2T axial magnetic field and measures momentum in the $|\eta| < 2.5$ region. The detector consists of three sub-detectors pixel, silicon microstrip and transition radiation trackers. The pixel tracker is situated closest to the beam pipe and is designed for a high-radiation environment. The highly granular detector makes on average 4 measurements per track. Following the pixel is the silicon microstrip detector which one average makes 4 measurements resulting in 4 space points. %Why?
The transition radiation tracker(TRT) surrounds the semiconductor detector in the barrel region(only barrel???) and consists of 4 mm drift straw tubes, generating trackick up to $|\eta| < 2.0 $. A track typically registers 36 hits per track in the TRT.

The calorimetry of ATLAS surrounds the solenoid and the ID with a coverage of $|\eta| < 4.9$. The electromagnetic calorimeter covers the barrel region($|\eta| < 1.475$) and the endcaps($1.375 < |\eta| < 3.2 $) with high-granularity lead/liquid-Argon(lAr) calorimetery. The hadronic section consists of tile calorimetry in the barrel and extended barrel regions($|\eta| < 1.7 $), the tile calorimetry use scintillating tiles as the active material and steel as the absorber. Additionally, the covered region is extended using an end-cap calorimeter covering $1.5 < |\eta| < 3.2 $ with lAr as the active material and copper absorbers. Finally, a dense forward calorimeter coveres the forward region inside the hadronic end-cap up to $|\eta| < 4.9 $.

The muon spectrometer covers the outermost parts of the ATLAS detector, the deflection of the muon tracks by the magnet system is use for the $p_T$ meaurements. The muon spectrometer consist of four sub-systems in three layers(cylindrical around beam axis in the barrel and perpendicular to the beam axis in the end-cap): Monitored Drift Tubes(MDT) and Cathode Strip Chambers(CSC) for precision tracking and Resistive Plate Chambers(RPC) and Thin Gap Chambers(TGC) for triggering.