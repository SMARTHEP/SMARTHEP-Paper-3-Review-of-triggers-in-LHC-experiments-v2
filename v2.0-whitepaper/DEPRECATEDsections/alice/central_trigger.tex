\subsection{ALICE Central Trigger System}

In Run 3, the ALICE experiment will run at significantly higher interaction rates than in Run 1 or Run 2, e.g. an increase of p-p interaction rates from \SI{2}{\kilo\hertz} to \SI{1}{\mega\hertz}. Additionally, many subdetectors have been upgraded to read out data continuously. These changes pose two challenges: of the synchronisation of subdetector readout across the detector, and of rapid transfers of large amounts of data. The CTS was redeveloped from the Run 2 Central Trigger Processor to handle these challenges. By delegating reconstruction and data reduction tasks to the HLT, the CTS maintains a throughput suitable for ensuring synchronised data readout collection.

The CTS synchronises data by subdividing the readout data into HeartBeat (HB) frames of approximately 1 LHC orbit period ($\approx\SI{88.92}{\micro\second}$). The status of each subdetector in a given HB frame are collated to form a Global HB Map. A HB decision is made on each HB frame by requiring an acknowledgement flag in all readout units being read out (up to 441 for the full ALICE detector), calculated as the logical and over a mask of the Global HB map. Each HB decision is transmitted asynchronously to the First Level Processor, to instruct it whether to keep any data received during a given HB frame.

Whilst many subdetectors (including the TPC) were upgraded to read out data continuously in Run 3, continuous readout is not possible for several subdetectors, e.g. the Transition Radiation Detector. Such subdetector are operated on a triggered basis and hence excluded from the HB decision calculation, instead operated independently with triggered data combined with continuous readout data at a later stage. The triggered detectors will use the existing RD12 TTC protocol developed for Run 2 operation, with the option to run at 2 or 3 trigger latency levels.