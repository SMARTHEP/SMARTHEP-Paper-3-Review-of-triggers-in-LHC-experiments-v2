\subsection{CMS Particle Flow algorithm} \label{sec:Algorithms}

An ever-increasing collection of algorithms for reconstruction of events and constituent objects are developed and adopted by collider physics experiments. One example is Particle Flow (PF)~\cite{sirunyan2017pflowcms} that combines measurements from the various sub-detectors in the ATLAS and CMS sub-detectors to produce particle candidates from the entire event. The general PF process can be briefly summarised as follows: the full detector output is used to describe the global collision event, identifying several basic elements individually and iteratively clustering them together into more complex composite physics objects~\cite{CMS:2020uim,CMS:2018rym,CMS:2014pgm}. The reconstructed physics objects are used to build electrons, muons, tau leptons, photons, jets, missing transverse momentum and other physics objects~\cite{CMS:2018jrd,CMS:2016lmd,CMS:2019ctu}. The physics performance is ameliorated by the combination of measurements from the different sub-detectors which achieve optimal accuracy in different regions of phase space. For example, the best momentum resolution is attained at low transverse momenta in the inner detector tracker and at high transverse momenta in the calorimeter. The composition of the tracks and calorimeter clusters (particle flow objects (PFOs)) are used in jet reconstruction to produce PFjets .

The PF procedure is widely-used in CMS reconstruction in both the trigger and offline analyses. The CMS detector design is well-suited to the use of PF reconstruction: a highly-segmented tracker, fine-grained electromagnetic calorimeter, hermetic hadron calorimeter, strong magnetic field and excellent muon spectrometer provide full, high quality coverage~\cite{sirunyan2017pflowcms}. The ATLAS experiment has expanded its usage of PF in the trigger for Run 3 to include hadronic jet reconstruction. The improvement in resolution is less pronounced in ATLAS due to the excellent energy resolution of the calorimeters and jets produced using only calorimeter clusters, however there is significant improvement in the rejection of pile-up (collisions in addition to the collision of interest occurring within the same detector sensitivity window)~\cite{ATLASTriggerRun3,ATLASJetPFlow}. 

% a widely-used reconstruction procedure in CMS offline analyses, now considered the ``baseline" for object reconstruction in the CMS HLT. The purpose of the PF algorithm is to reconstruct and identify all particles involved in a collision by combining and correlating information from all sub-detectors. The CMS detector design is well-suited to the use of PF reconstruction: a highly-segmented tracker, fine-grained electromagnetic calorimeter, hermetic hadron calorimeter, strong magnetic field and excellent muon spectrometer provide full, high quality coverage~\cite{sirunyan2017pflowcms}. The ATLAS experiment implements PF algorithms offline and partly online. The ATLAS detector gains less from the approach due to its excellent energy resolution in the calorimeters but for low-momentum resolution and pile-up rejection the improvement is substantial~\cite{ATLASTriggerRun3,ATLASJetPFlow}.

\subsection{ML b-jet identification algorithms in ATLAS} \label{sec:Algorithms}

More recently, machine-learning-based algorithms have emerged as options for fast reconstruction. In particular, Graph Neural Networks (GNNs), a class of machine learning models designed to operate on graph-structured data, have seen increasing adoption across the LHC experiments. For example, GNNs are used in the ATLAS Run~3 $b$-jet trigger, where tracks of common vertices are grouped and predictions around jet origin are made~\cite{ATLASTriggerRun3}. 

GNNs propagate information through the nodes and edges of a graph, capturing complex relationships and dependencies within the data. GNNs can learn to perform various tasks, such as node classification, link prediction, and graph classification, for example the cluster of hits and reconstuction of showers in the LHCb calorimeter systems \cite{canudas2022graph}.

\todo[inline]{Need an example of an ALICE algorithm to complete the set of examples.}
