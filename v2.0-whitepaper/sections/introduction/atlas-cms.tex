\subsection{Trigger systems of the ATLAS and CMS experiments}

The two general-purpose LHC experiments, ATLAS and CMS, have similar, broad physics programmes~\cite{ATLASMachine,collaboration2008cms}. 
These programmes include both measurements of the Standard Model and searches for new physics: indirectly, by probing the nature of known particles at the precision frontier, and directly (e.g. through searches for new particles and dark matter candidates~\cite{snowmass-darkmatter}). 
Both experiments consist of concentric layered subdetectors; namely inner detectors for charged particle tracking, calorimeters for measurement of hadron/electromagnetically interacting particle energies and positions, and muon spectrometers for the identification and precise measurement of muons. 
Due to the large acceptance and high granularity of the subdetectors used, the experiments contain a large number of detector channels, resulting in event sizes of the order of  ${\sim}\SI{1}{\mega\byte}$. 
ATLAS and CMS implement similar two-tier trigger systems with a Level-1 (L1) hardware-based trigger and a software-based high-level trigger (HLT). In Run~2 and Run~3, the triggers of both experiments reduced the initial \SI{40}{\mega\hertz} bunch crossing rate to an L1 acceptance rate of \SI{100}{\kilo\hertz}, reduced further to a final HLT output rate of ${\sim}\SI{1}{\kilo\hertz}$. For average event sizes ranging from ${\sim}\SI{500}{\kilo\byte}$~\cite{ATLASRun3EventBuilder} to ${\sim}\SI{2}{\mega\byte}$~\cite{cmsRun3EventBuilder}, this corresponds to a HLT output bandwidth of $\mathcal{O}\left(\SI{1}{\giga\byte\per\second}\right)$.
