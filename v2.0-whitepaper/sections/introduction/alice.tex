\subsection{Trigger system of the ALICE experiment}
The ALICE experiment is dedicated to the study of heavy-ion collisions at the LHC, with a focus on studies of quantum chromodynamics in energy-dense environments (e.g.,  quark-gluon plasma)~\cite{alice-performance-paper-run1}. To study such environments, ALICE studies p-p, p-Pb and Pb-Pb collisions at rates of \SI{1}{\mega\hertz}, \SI{500}{\kilo\hertz} and \SI{50}{\kilo\hertz}, respectively \cite{alice-trigger-run3}. Heavy ion collisions result in a very high multiplicity of particles, with ${\sim}\SI{700}{\mega\byte}$ of raw data per collision event collected by the ALICE experiment \cite{alice-rta-trigger}. The ALICE detector is barrel-shaped, containing concentric particle tracking and identification systems, and a forward muon spectrometer. At the core of the barrel is a Time Projection Chamber (TPC) vital to tracking performance, contributing the majority of the event size (\SI{95.3}{\percent} and \SI{91.1}{\percent} of total data volume in Pb-Pb and p-p collisions, respectively).

The ALICE trigger was upgraded ahead of Run~3 to facilitate continuous readout of subdetectors at the collision rates listed above. The Central Trigger System (CTS), is responsible for the synchronisation of raw detector readout. The CTS transmits aggregated data and trigger signals to the HLT. The HLT then performs event reconstruction, data volume reduction and subdetector calibration, processing data at a maximum rate of \SI{48}{\giga\byte\per\second} to achieve an output throughput of up to \SI{12}{\giga\byte\per\second}~\cite{alice-rta-trigger}.
