\section{Conclusion}

The trigger systems at the major LHC experiments have undergone a continuous evolution throughout their operation in Run~1 and Run~2. Significant upgrades have been implemented and proposed to address the challenges posed by the current Run~3 and future high luminosity conditions. The triggers now take advantage of modern hardware developments in the sub-detector readouts and novel data processing software including contemporary machine learning developments. The result today is a more robust, flexible and efficient trigger system, better able to accommodate future requirements.

In Run~3, the major LHC experiments leverage their experience from Run~1 and Run~2 to tackle increased data challenges. In particular, upgrades to these experiments and their respective trigger systems take advantage of recent advances in hardware technologies, with software frameworks redesigned to better capitalise on such upgrades. New and optimised software frameworks also provide implementations for faster and more capable algorithms for reconstruction, selection and data manipulation. Overall, this experience will be integrated into the plans for Run-4, also taking advantage of new detectors and new trigger hardware capabilities. 
