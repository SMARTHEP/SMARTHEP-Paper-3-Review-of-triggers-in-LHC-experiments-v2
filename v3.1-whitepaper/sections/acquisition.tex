\section{Data processing and selections in the lower-level triggers}
% Introduce and group ATLAS+CMS+ALICE/LHCb
The first stage of any trigger system requires the readout of reduced granular detector information (e.g., energy deposits in a calorimeter/tracker) and processing of this information. 
This processing has two main steps. First, information from the subdetectors must be synchronised to ensure that the corresponding data are correctly associated to an event.  
Subsequently, the data volume needs to be reduced through a combination of basic compression techniques (e.g., zero suppression) and selections. 
This is the common task of a low-level trigger - once the event is selected, the detector (or subdetector) is read out. 
This section covers only ATLAS, CMS and ALICE since LHCb no longer employs a low-level trigger: detector information is continuously read out (with compression applied upon readout) and passed to the event builder network of HLT1~\cite{LHCb:2023hlw}.
This approach is better suited to LHCb, with smaller event sizes and a non-hermetic detector geometry (readout cables sit outside detector acceptance).

%The smaller event size and non-hermetic layout(readout cables sit outside detector acceptance) means this change is more achievable for LHCb than ATLAS and CMS.
%\todo[inline]{Waiting for feedback on the sentence above} 
%Large ATLAS on tiered trig sys, similar structure in CMS, slight difference in ALICE, then tiny LHCb caveat/sentence to be discussed later

%ATLAS+CMS+ALICE
The initial hardware-based L1 triggers of ATLAS~\cite{ATLASRun3Detector} and CMS~\cite{CMS:run3-detector} both reduce the rate of data down to a maximum detector read out limit of \SI{100}{\kilo\hertz}, within a latency of \SI{2.5}{\micro\second} at ATLAS and \SI{4}{\micro\second} at CMS. 
The systems consist of custom Algorithm Specific Integrated Chips (ASIC)- and Field Programmable Gate Arrays (FPGA)-based electronics~\cite{asics-fpgas}. 
For Run 3, the ATLAS and CMS L1 triggers use reduced granular information from the calorimeter and muon systems (and the combination of the two) to perform coarse selections~\cite{ATLASTriggerRun3, cms2020performance}. 
Upon an event passing this selection, the L1 directs the readout hardware of each subdetector to process the surviving data, which is stored in associated local buffers. 
Information from each subdetector is  then readout, processed and combined within a central readout system where events are assembled. 
It is buffered in the HLT farm until requested for processing by the HLT. 
The data acquisition (DAQ) systems of both experiments have evolved to use consumer network and computing hardware downstream of custom on-board electronics. 
This setup simplifies the readout in complexity, maintenance cost and upgrade capabilities. 
%At ATLAS the Front-End LInk eXchange (FELIX)~\cite{ATLAS:FELIX} readout boards, responsible for the interface between commercial and custom hardware, have been partially implemented in Run 3 and will be fully implemented for HL-LHC.
%\todo[inline]{Consider removing the above sentence because we don't have equivalents for ALICE and CMS}


In Run~3, several of the ALICE subdetectors have been upgraded to read out data continuously. The CTS synchronises data, subdividing readout into HeartBeat (HB) frames of approximately 1 LHC orbit period ($\sim\SI{88.92}{\micro\second}$). An HB frame is only kept if all relevant readout units (up to 441 in the entire detector) can be read out. Each HB decision is transmitted asynchronously to the First Level Processor, instructing it on what data to keep in a given HB frame. Whilst many subdetectors (including the TPC) were upgraded to read out data continuously in Run~3, continuous readout is not possible for some subdetectors, e.g.,  the Transition Radiation Detector. Such subdetectors are operated on a triggered basis and are hence excluded from the HB decision calculation, instead using the existing RD12 TTC protocol developed for Run~2 operation. Triggered subdetectors thus operate independently, with triggered data combined with continuous readout data at a later stage~\cite{alice-trigger-run3}.

