\subsection{The ALICE experiment}
The detector of the ALICE experiment is structured around a large central barrel. A 0.5 T solenoid magnet surrounds the barrel, within which (from outermost to innermost) are the Photon Spectrometer (PHOS), Electromagnetic Calorimeter (EMCAL), Time-Of-Flight (TOF) detector, Transition Radiation Detector (TRD), Time Projection Chamber (TPC), V0 and T0 fast-interaction detectors and the Inner Tracking System (ITS). The muon spectrometer is placed forward of the barrel, separated by a  dipole magnet. Additional subdetectors, the ALICE Diffractive (AD) detector and Zero Degree Calorimeter (ZDC) are placed in the extreme forward and backward positions.

Operating in such a high-multiplicity environment, per collision data sizes in ALICE are typically much larger than in other LHC experiments, with an average of $700~\mathrm{mB}$ produced in each collision event, though this can be reduced significantly by means of a zero-suppression algorithm. However, collision events take pace less frequently than at other experiments, with event rates of up to $2~\mathrm{kHz}$ in p-p/p-Pb collisions and $1~\mathrm{kHz}$ in Pb-Pb collisions, rather than at the $40~\mathrm{MHz}$ conventional at other LHC experiments.

ALICE is unique in a number of ways. Firstly, in that the majority of the detector readout (which can reach approximately $50~\mathrm{GBs}$) arises from one subdetector, the TPC, responsible for a 91.1~\% and 95.3~\% of total data volume in Pb-Pb and p-p collisions, respectively. Additionally, in Run 1 and Run 2, data was read out in a discrete manner, wherein events would only be taken when the trigger fired, though detector upgrades permitted continuous detector readout for Run 3.

