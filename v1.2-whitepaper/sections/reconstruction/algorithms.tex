\subsection{Algorithms} \label{sec:Algorithms}

An ever-increasing collection of algorithms for reconstruction of events and constituent objects are developed and adopted by collider physics experiments. One such example is Particle Flow (PF)~\cite{sirunyan2017pflowcms}, a widely-used reconstruction procedure in CMS offline analyses, now considered the ``baseline" for object reconstruction in the CMS HLT. The purpose of the PF algorithm is to reconstruct and identify all particles involved in a collision by combining and correlating information from all sub-detectors. The CMS detector design is well-suited to the use of PF reconstruction: a highly-segmented tracker, fine-grained electromagnetic calorimeter, hermetic hadron calorimeter, strong magnetic field and excellent muon spectrometer provide full, high quality coverage~\cite{sirunyan2017pflowcms}. The ATLAS experiment implements PF algorithms offline and partly online, the ATLAS detector gains less from the approach due to the strong energy resolution in the calorimeters but for low momentum resolution and pile-up rejection the improvement is substantial~\cite{ATLASTriggerRun3,ATLASJetPFlow}.

The overall PF process can be briefly summarised as follows: the full detector output is used to describe the global collision event, identifying several basic elements individually and iteratively clustering them together into more complex composite physics objects~\cite{CMS:2020uim,CMS:2018rym,CMS:2014pgm}. The reconstructed physics objects are used to build electrons, muons, tau leptons, photons, jets, missing transverse momentum and other physics objects~\cite{CMS:2018jrd,CMS:2016lmd,CMS:2019ctu}. The combination of measurements from the different sub-detectors allows access to the precision of each sub-system within their optimal respective phase spaces (i.e.,  using tracker information at low momenta and calorimeter information at higher momenta calo. better).


More recently, machine-learning-based algorithms have emerged as options for fast reconstruction. In particular, Graph Neural Networks (GNNs), a class of machine learning models designed to operate on graph-structured data, have seen increasing adoption across the LHC experiments, e.g.,  in the ATLAS Run~3 trigger~\cite{que2023llgnn}. GNNs propagate information through the nodes and edges of a graph, capturing complex relationships and dependencies within the data. GNNs can learn to perform various tasks, such as node classification, link prediction, and graph classification.


Another class of algorithms that is widely used at LHC experiments is Kalman filters (KFs). KFs are recursive algorithms used for state estimation and data fusion, with a well-established position as a key reconstruction tool~\cite{kalman-track}, and are particularly well-suited for problems involving dynamic (i.e.,  time-dependent) systems and noisy measurements. KFs combine information from previous state estimates and new measurements to provide an optimal estimate of the current state, taking into account both the dynamics of the system and the uncertainties associated with each measurement~\cite{FRUHWIRTH1987444}. KFs are employed across the major LHC experiments~\cite{Belikov:2003yr,ATLAS:tracking,CMS:tracking,LHCb_Allen_GPU} for tracking and identification of particle trajectories.