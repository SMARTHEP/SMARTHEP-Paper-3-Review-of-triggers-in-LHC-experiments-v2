\subsection{Reconstruction procedure}

In this section we describe the LHCb reconstruction procedure as it provides a clear example of the processes involved in reconstructing event objects. Equivalent processes take place in the reconstruction chains of the other LHC experiments with variations due to the specific detector architecture.
% The LHCb reconstruction procedure provides a clear example of the processes involved in reconstructing events and event objects. 

In HLT1, information from the tracking—Vertex Locator (VELO), Upstream Tracker (UT) and Scintillating Fibre (SciFi) Tracker are used to perform partial event reconstruction using a dedicated CUDA-based software framework~\cite{LHCb_Allen_GPU}. For each detector, many of the following processes must be performed: decoding of input into the global coordinate system; clustering of hits from a given particle candidate; combination of hits/clusters to form particle trajectories; fitting of track model to track candidates; vertex-finding between tracks.

In the VELO, reconstruction begins with the clustering of hits on each silicon plane, with a bit mask-based clustering algorithm operating in parallel across the local regions of each cluster. Straight-line tracks are then reconstructed, starting with seeds of three hits from consecutive silicon planes, extended to the remaining layers and fit with a simple Kalman filter (discussed further in Section~\ref{sec:Algorithms}). Finally, primary vertex (PV) candidates are identified and matched to the tracks. Rather than mapping tracks one-to-one with PV candidates, each track receives a per-candidate weight to enable parallel computation. 

No clustering is performed in the UT detector as most events cause only a single-fired silicon strip. UT hits are instead assigned to extrapolated VELO tracks using a minimum momentum cut-off, with the track momentum calculated from the curvature between the straight-line VELO tracks and UT hits.

The clustering of SciFi hits is performed directly on the readout board so that the algorithm only needs to decode raw data and perform track reconstruction. Tracks passing the VELO and UT are extrapolated to create a search window in the SciFi, where seeds of three hits in different layers are formed. A $\chi^2$ fit is performed on the seeds, and the best seeds are extrapolated to the remainder of the SciFi layers. Since the discrimination power of the three initial hits is limited, several seeds are extrapolated for each UT track, performing additional fits to select the best track per UT track. Hits in the muon systems are matched to extrapolated SciFi tracks according to the track parameters obtained in the previous steps~\cite{LHCb:2023hlw, LHCb_Allen_GPU}.