\section{Alignment and calibration of detectors}

Accurate alignment and calibration of detectors are crucial to the quality of event reconstruction, ensuring faithful decoding of raw detector information to reconstructible objects. This is particularly important for offline-quality reconstruction, where the final reconstructed objects are later used in physics analysis. Alignment and calibration are typically implemented together (where alignment often refers to tracking subdetectors and calibration to non-tracking subdetectors, e.g., calorimeters), performing a minimisation of parameters describing the physical offsets experienced in a given subdetector, e.g., residuals between the expected position of a track and the measured position. Beyond offline-quality reconstruction, alignment and calibration techniques are also applied when partial reconstructions require more accurate knowledge of an object. The real-time approaches of LHCb and ALICE in Run~3 typify this and are thus the focus of this chapter. At ATLAS and CMS, the complex detector nature prevents full alignment and calibration in real time.

% LHCb
%\subsection{Real-time detector alignment in LHCb}
The large timing budget of the HLT stages, provided by the buffers, enables the real-time alignment and calibration of the LHCb detector, taking advantage of the multi-core infrastructure of the EFF. Performing calibration at trigger level ensures offline-quality reconstruction, enabling trigger selection with high signal efficiencies. This calibration is typically separated into the alignment of the VELO, UT/SciFi, muon stations and RICH mirrors, and the calibration of the ECAL and RICH. A dedicated data sample\footnote{LHCb employs an alignment trigger configuration (alongside the selection triggers), with the selected events saved to a dedicated stream described further in Section~\ref{sec:data}.} is collected, from which the alignment framework calculates updated alignment constants at regular intervals, i.e., per fill or run, as soon as sufficient data has been collected for precise calculation of the constants.

The alignment constants for the tracking detectors are determined by minimising track reconstruction parameters with respect to the degrees of freedom of each alignable detector element, i.e., translations and rotations in each spatial dimension. The UT and SciFi are also aligned using reconstructed tracks traversing the entire tracking system to achieve high track-momentum resolution \cite{Reiss:2846414}. The ECAL is calibrated by analysing a specific mass distribution in each ECAL cell, providing a mass shift from the known position which can be applied as an adjustment of the photomultiplier tubes after each fill. The RICH is aligned and calibrated on a per-run basis, comparing the distribution of hits and Cherenkov angles to anticipated distributions to generate alignment and calibration constants~\cite{LHCb:RICH_AlignCalib}.
%the $\pi^0\rightarrow \gamma\gamma$ 

% ALICE
%\subsection{Real-time TPC alignment in ALICE}
At ALICE, as a consequence of upgrades for continuous readout in Run~3, the TPC requires real-time calibration to correct for space-charge density distortions and fluctuations in such distortions: space-charge distortions are particularly problematic in high pile-up scenarios and fluctuations in space-charge distortions vary on a scale of \SI{10}{\milli\second}. Average space-charge distortions are corrected with the aid of relevant subdetectors; however, the calibration of average space-charge distortions requires $\mathcal{O}\left(\mathrm{minutes}\right)$ of data collection to form a reliable correction map. Convolutional neural networks (CNNs) are employed to predict the fast-varying fluctuations in space-charge distortion, obtaining a reliable calibration at the fluctuation time scale. The CNN models run GPUs directly in the HLT farm, resulting in a significant speedup of the TPC calibration.
