\section{Conclusion}

The trigger systems at the major LHC experiments have undergone a continual evolution throughout their operation in Run~1 and Run~2. Significant upgrades have been both realised and proposed in the face of the present Run~3 and future high luminosity conditions. The triggers now take advantage of modern hardware developments in the sub-detector readouts and novel data processing software including contemporary/novel machine learning developments. The result today is a more robust, flexible and efficient trigger system, better able to accommodate future requirements.

In Run~3, the major LHC experiments leverage their experience from Run~1 and Run~2 to tackle increased data challenges. In particular, upgrades to these experiments and their respective trigger systems take advantage of recent advances in hardware technologies, with software frameworks redesigned to better capitalise on such upgrades. New software frameworks also provide implementations for faster and more capable algorithms for reconstruction, selection and data manipulation.
