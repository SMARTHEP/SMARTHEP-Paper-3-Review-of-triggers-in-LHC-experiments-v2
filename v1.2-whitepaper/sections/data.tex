\section{Handling of data}
\label{sec:data}

Reconstructed detector events are typically complex data objects described by many parameters. The size of events presents a storage challenge as it can become infeasible to write all reconstructed data to memory at the output rate of the high-level trigger. To overcome this challenge, two key strategies are employed to reduce the size of the reconstructed objects. Firstly, the amount of information can be reduced, by selecting only the minimal amount of parameters necessary to sufficiently describe the event to be propagated to storage. Secondly, the object information itself can be compressed, maintaining most or all of the original information while significantly reducing the size of the data being written.

% Streaming
%\subsubsection{Data scouting at ATLAS (TLA) and CMS}
An example of the first approach is the CMS data scouting stream or TLA in ATLAS, wherein events are stored at higher rates with smaller event content - storing simply the online objects, bypassing offline reconstruction entirely. In addition, a parking stream or delayed stream saves event data without running offline reconstruction algorithms, with the intent of processing said events during shutdown periods when computing resources are less constrained. Monitoring and calibration streams are also able to make use of their small event size to store more events, with a typical calibration stream event size of ${\sim}\SI{1.5}{\kilo\byte}$, in comparison to an event with full detector readout at around \SI{2}{\mega\byte}~\cite{cms2023development}. However, the additional information of the latter event model allows a wide range of track reconstruction algorithms to be applied for the identification of b-jets, lepton isolation and the mitigation of pile-up vertices~\cite{mia2014trackingcms,tosi2016trackingcms}. 

\begin{figure}[h!]
    \centering
    \includegraphics[width=0.55\linewidth]{images/atlas/ATLAS-TLA.png}
    \caption{Comparison of offline and trigger-level jet objects in ATLAS dijet spectra from Ref.~\cite{ATLAS:2018qto}. Trigger-level jets provide low $m_{jj}$ coverage without a requirement for a prescale correction.}
    \label{fig:trigger-level-efficiencies}
\end{figure}

%\subsubsection{The LHCb Turbo model}
Another example of such an approach is the LHCb Turbo model, a reduced-persistency event model, in which only the objects relevant to an HLT2 trigger decision (and any primary vertices) are persisted and saved to a dedicated stream~\cite{Aaij:2016rxn}. For inclusive triggers, which select events across many signals, many objects are still required. To circumvent this, inclusive trigger events are saved to a FULL stream, which can later be reselected by dedicated exclusive lines to extract only the relevant event objects. Many trigger-level objects may be required which are beyond the standard objects of a given decay, for example for flavour tagging. In such cases, selective persistency is applied, whereby an event is saved in the Turbo model with specified additional information, e.g., other tracks arising from a primary vertex, objects contained in a cone around the candidate, etc., A calibration stream (TurCaL) is a special use case of the selective persistence model, wherein only events dedicated to detector alignment and calibration are persisted~\cite{Aaij:2019uij}. % Discuss alignment further <- <- <-


%% COMPRESSION
% Baler? GPUs in ALICE, ML in CMS. Retina?

%model follows that for the fixed bandwidth, 10 GB/s due to offline storage resource constrains, the event rate can be increased, and therefore the physics sensitivity of the experiment, by reducing the event size saving only the ``interesting” part of the event. 
