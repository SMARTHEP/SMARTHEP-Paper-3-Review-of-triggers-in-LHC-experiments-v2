\subsection{Trigger system of the LHCb experiment}

The LHCb experiment is a heavy-flavour experiment operating in the forward region, searching for new physics through precision studies of the properties of heavy-flavour decays, in particular CP- and flavour-violation. Following the upgrade of the detector prior to Run~3, the LHCb detector operates as a general-purpose forward detector. The LHCb trigger system was redesigned for Run~3, removing the low-level Level 0 (L0) hardware-based trigger previously employed in Runs 1 and 2. The simplistic cut on $p_{T}$ implemented in the L0 trigger could not discriminate between signal and background for hadronic signals, which would have resulted in a effeciency loss as seen in Fig. \ref{fig:LHCbL0TriggerYield}. As such, the LHCb trigger consists solely of a HLT, split between two stages: HLT1 and HLT2~\cite{Aaij:2019uij}. Following the removal of the L0 trigger, LHCb event readout has increased from \SI{1}{\mega\hertz} to \SI{30}{\mega\hertz}. During Run~2, HLT1 and HLT2 were decoupled to allow HLT1 to run synchronous to data-taking and HLT2 to run asynchronously, enabling  detector calibrations between the steps to improve reconstruction performance to offline quality~\cite{LHCb:Albrecht_2015}. 

\begin{figure}[h!]
    \centering
    \includegraphics[width=0.55\linewidth]{images/lhcb/LHCb-L0-yield.png}
    \caption{Trigger yield per mode of interest with the Run 2 trigger configuration from Ref.~\cite{LHCb:upgrade-piucci}. Any increase in luminosity from accelerator upgrades is suppressed by the L0 trigger in all modes but $\psi\phi$ (i.e., non-muonic modes).}
    \label{fig:LHCbL0TriggerYield}
\end{figure}

HLT1 was upgraded throughout Run 2 to perform a partial reconstruction of the full detector readout. To achieve this, the reconstruction algorithm was upgraded to be able to run on GPUs hosted on the same event-building servers that host the FPGA cards required to receive data from the detector at 30 MHz~\cite{LHCb_Allen_GPU}. Reconstructed, selected events are propagated to a buffer at an event rate of ${\sim}\SI{1}{\mega\hertz}$. HLT2, implemented as a CPU farm known as the Event Filter Farm (EFF), takes as input the most recent detector alignment and calibration, reconstructing events in full offline quality for detailed selection. This selection reduces the event rate to \SI{100}{\kilo\hertz}, corresponding to an output bandwidth of \SI{10}{\giga\byte\per\second}~\cite{lhcb_hlt2_storage_run3}.
